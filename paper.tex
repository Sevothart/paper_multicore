\documentclass[10pt, conference]{IEEEtran}

\usepackage[english]{babel}
\usepackage[utf8]{inputenc}
\usepackage[pdftex]{graphicx}
\usepackage{url}
\usepackage{listings}
\usepackage{subfigure}
\usepackage{multirow}
\usepackage[space]{cite}
\usepackage{siunitx}
\usepackage{algorithm}
\usepackage{algpseudocode}
\usepackage{microtype}
\usepackage{latexsym}
\usepackage{textcomp}

\usepackage{amssymb}
\usepackage{amsmath}
\usepackage{pictexwd}
\usepackage[absolute]{textpos}
\usepackage[htt]{hyphenat}

\usepackage{pgfplots}
\pgfplotsset{compat=1.11}

\sisetup{group-separator = {,}, group-digits = true}
\newcommand\sius[1]{\num[group-separator = {,}]{#1}\si{~\micro\second}}
\newcommand\sims[1]{\num[group-separator = {,}]{#1}\si{~\si{\milli\second}}}
\newcommand\sins[1]{\num[group-separator = {,}]{#1}\si{~\nano\second}}

\newcommand\GG[1]{$\Diamond$\footnote{Giovani: {#1}}}
\newcommand\LC[1]{\textleaf\footnote{Lucas: {#1}}}

%\interfootnotelinepenalty=500

\lstset{keywordstyle=\bfseries, flexiblecolumns=true, inputencoding=latin1}
\lstloadlanguages{[ANSI]C++}
\lstdefinestyle{prg} {basicstyle=\scriptsize, lineskip=-0.2ex, showspaces=false,breaklines=true,showstringspaces=false,numbers=left,numbersep=-7pt,frame=single,stepnumber=1}

\newcommand{\prg}[3][ht!]{
  \begin{figure}[#1]
      \lstinputlisting[language=C++,style=prg,basicstyle=\scriptsize]{fig/#2.cc}
    \caption{#3\label{prg:#2}}
  \end{figure}
}

\newcommand{\tab}[3][ph]{
  \begin{table}[#1]
    {\centering\small\textsf{\input{fig/#2.tab}}\par}
    \caption{#3}
  \label{tab:#2}
  \end{table}
}

\newcommand{\fig}[4][ht!]{
  \begin{figure}[#1]
    {\centering{\includegraphics[#4]{fig/#2}}\par}
    \caption{#3}
    \label{fig:#2}
  \end{figure}
}

\newcommand{\Fig}[4][!htb]{
  \begin{figure*}[#1]
    {\centering{\includegraphics[#4]{fig/#2}}\par}
    \caption{#3}
    \label{fig:#2}
  \end{figure*}
}

\newcommand{\Tau}{\mathrm{T}}
\newcommand{\ms}{\textmu{}s}

\newcommand{\AlgInput}{\textbf{Input: }}
\newcommand{\AlgOuput}{\textbf{Output: }}
\newtheorem{theorem}{Theorem}[section]
\newtheorem{lemma}[theorem]{Lemma}
\newtheorem{proposition}[theorem]{Proposition}
\newtheorem{corollary}[theorem]{Corollary}
\newtheorem{definition}[theorem]{Definition}


\makeatletter
\renewcommand{\ALG@beginalgorithmic}{\scriptsize}
\makeatother

\IEEEoverridecommandlockouts

\begin{document}
%
% paper title
% can use linebreaks \\ within to get better formatting as desired
\title{On the Design and 
Implementation of Uniprocessor
Real-Time Resource Access 
Protocols\IEEEauthorrefmark{1}\thanks{\IEEEauthorrefmark{1}Work 
funded by CNPq/PIBIC.}}

% author names and affiliations
% use a multiple column layout for up to three different
% affiliations
\author{\IEEEauthorblockN{Lucas Pires Camargo}
\IEEEauthorblockA{Software/Hardware Integration Lab\\
Federal University of Santa Catarina\\
Joinville, Santa Catarina, Brasil\\
Email: camargo@lisha.ufsc.br}
\and
\IEEEauthorblockA{Giovani Gracioli\\
Software/Hardware Integration Labb\\
Federal University of Santa Catarina\\
Joinville, Santa Catarina, Brasil\\
Email: giovani@lisha.ufsc.br}}

\maketitle
\thispagestyle{plain}
\pagestyle{plain}

\begin{abstract}
Real-time operating systems (RTOS) should support resource access protocols 
to bound the maximum delay incurred by priority inversions. The implementation 
of such protocols must be lightweight, because its performance affects the 
system schedulability. In this paper, we present an object-oriented design of 
real-time resource access protocols for uniprocessors aiming at reducing the 
run-time overhead and increasing code re-usability. We implement the proposed 
design in an RTOS and measure the memory footprint and run-time overhead of the 
implementation. By applying the obtained overhead into the 
schedulability of several generated task sets and four protocols, our results 
indicate that proper implementation of resource access protocols has a small 
impact on the schedulability of real-time tasks.
\end{abstract}

\begin{IEEEkeywords}
Real-time resource access protocols, real-time operating systems, 
priority ceiling protocol, priority inheritance protocol, stack resource 
policy
\end{IEEEkeywords}

\IEEEpeerreviewmaketitle

\section{Introduction}
\label{sec:intro}

Any concurrent operating system (OS) must offer some kind of protocol to 
control the access to shared resources (\emph{i.e.,} a piece of code, such as 
data structures, I/O devices, buffers, and so on) by competing tasks, thus 
ensuring \emph{mutual exclusion} in their respective critical 
sections~\cite{Liu:2000,Buttazzo:2011}. 

Typically, mutual exclusion is guaranteed by the use of binary semaphores, such 
as mutexes and suspension-based locks~\cite{Yang:2015}. Therefore, a task 
willing to enter a critical section must wait until another task, which holds 
the resource, exits the critical section. This is accomplished by calling the 
semaphore operations \emph{p} (or wait) and \emph{v} (or signal) before 
entering and after leaving each critical section, respectively.

Real-time embedded applications also use semaphores to synchronize access to 
shared resources. However, specific resource access protocols are required to 
avoid unbounded priority inversions~\cite{Liu:2000,Buttazzo:2011,Yang:2015}. 
For instance, consider a higher-priority task $\tau_a$ and a lower-priority 
task $\tau_b$, sharing a resource $R_1$. It might happen that $\tau_b$ is 
holding the resource $R_1$, but is preempted by $\tau_a$. Then, $\tau_a$ 
executes until it tries to enter the critical section. However, $\tau_a$ cannot 
continue, because the resource is already in use by $\tau_b$. Thus, a 
higher-priority task ($\tau_a$) is blocked by a lower-priority task ($\tau_b$) 
which can cause unpredictable delays if not properly handled.

In fact, priority inversion has caused many problems in real applications. For 
example, the mars pathfinder spacecraft reseted several times after landing 
Mars on July $4^{th}$, 1997, resulting in significant delays in capturing 
scientific data~\cite{Jones:1997,Reeves:1997}. This resetting was caused by the 
overran of a lower-priority task due to priority inversion, which triggered a 
watchdog timer. Fortunately, the developers were able to patch the spacecraft 
remotely to solve the problem.  

Several real-time resource protocols have been proposed to bound priority 
inversion and take the blocking time into consideration when performing 
schedulability analyses. These protocols can be divided in those for static 
schedulers (when task priorities do not change) and for dynamic schedulers 
(when task priorities may change during execution)\footnote{Note that in this 
work we only consider resource access protocols for Uniprocessors.}. Ceiling- 
and priority inheritance-based protocols~\cite{Sha:1990} are typically used in 
static scheduling, while the stack resource protocol (SRP)~\cite{Baker:1991} is 
typically used in dynamic scheduling.

In this work, we present a design and implementation of resource access 
protocols for uniprocessors, considering both static and dynamic scheduling 
approaches. More specifically, we present a design for the Priority Inheritance 
Protocol (PIP)~\cite{Sha:1990}, Priority Ceiling Protocol 
(PCP)~\cite{Sha:1990}, Immediate Priority Ceiling Protocol (IPCP), and 
SRP~\cite{Baker:1991}. Our design uses object-oriented techniques to 
maximize the software reuse and minimize the run-time overhead. We have 
implemented the proposed software design in a real-time operating system (RTOS) 
and evaluated the memory footprint and run-time overhead in a microcontrolled 
platform. Our results indicate that the proposed design allows software reuse 
and low overhead, providing schedulability ratios close to the theoretical 
ones. 
In summary, we make the following contributions in this paper:

\begin{itemize}
 \item We propose an object-oriented design for uniprocessor real-time resource 
access protocols. The focus is on software reuse and low overhead;
 \item We implement the proposed design in an RTOS and evaluate the memory 
footprint of this implementation as well as its run-time overhead on a 
microcontrolled platform. The maximum obtained overhead is less than 
20~\si{\micro\second}, considering both \emph{p} and \emph{v} operations;
 \item A comparison in terms of task set schedulability ratio among the 
protocols, considering their run-time overheads. Our results indicate that due 
to the low overhead, the schedulability ratio remains close to theoretical 
bounds, proving the efficiency of the proposed design.
\end{itemize}

The remainder of this paper is organized as follows. Section~\ref{sec:back} 
presents the system model and reviews the resource sharing protocols used in 
this work. Section~\ref{sec:des} shows the proposed design and implementation 
of the protocols. Section~\ref{sec:eval} evaluates the proposed design in an 
real-time operating system and real hardware platform. Section~\ref{sec:rel} 
presents the related work. Finally, Section~\ref{sec:conc} concludes the paper.

\section{System Model and Background}
\label{sec:back}

In this  work we  consider the periodic task model, in which a  task set  
$\tau$ is composed  of \textit{n}  implicit-deadline  tasks, $\tau$ = 
\{$\tau_1$, ... ,$\tau_n$\}, running on a single-processor. Each task  
$\tau_{i}$, where \textit{i} $\leq$  \textit{n} and \textit{i} $\geq$ $1$, has a 
 period \textit{$p_{i}$} and  a worst-case execution time (WCET) 
\textit{$e_{i}$}. A task $\tau_{i}$  releases a job at every \textit{$p_{i}$} 
time units. $r_i^j$ denotes the release time of the $j^{th}$ job of $\tau_{i}$, 
named $\tau_i^j$. The relative deadline of the task $\tau_i$ is equal to its 
period: $d_i$ $=$ $p_i$. The relation \textit{$e_{i}$}/\textit{$p_{i}$} defines 
the utilization of a task $\tau_{i}$, called $u_i$. The sum of all tasks' 
utilizations defines the total system utilization ($\sum_{i=1}^n u_i$). We 
assume that tasks suspend only to wait for a lock. We also assume that the task 
set is scheduled either by the Rate-Monotonic (RM) scheduling, when
fixed-priority (FP) policy is used, or by the Earliest Deadline First (EDF), 
when dynamic-priority policy is used. At run-time, resource access protocols 
may temporarily raised the tasks priorities. 

Tasks share $n_r$ serially-reusable resources, $R_1$,$\ldots$,$R_{n_r}$. 
$N_{i,q}$ denotes the maximum number of times that a job of $\tau_i$ accesses 
$R_q$. $L_{i,q}$ denotes the maximum critical section length required by 
$\tau_i$ when it uses $R_q$. The priority ceiling $C_q$ is equal to the 
priority of the highest-priority task that uses $R_q$. 

\subsection{Resource Access Protocols}

This sections presents an overview of the real-time synchronization protocols 
implemented in this work. The protocols are the Priority Inheritance Protocol 
(PIP), the Priority Ceiling Protocol (PCP), the Immediate Priority Ceiling 
Protocol (IPCP), and the Stack Resource Policy (SRP). For a complete overview 
of such protocols, please refer to~\cite{Liu:2000,Buttazzo:2011}.

PIP is a classic mechanism for sharing resources in a single-processor with
FP scheduling. It aims at avoiding priority inversion by 
elevating the priority of the tasks that hold a resource, when there are higher 
priority tasks waiting on the same resource. The priority of the running task 
is always the maximum priority of the tasks blocked on the resource, if it is 
higher that the original priority.

PCP is another classic algorithm for controlling priority inversion and 
bounding blocking time for a task set with shared resources. It differs from 
PIP in the sense that every resource has a priority ceiling $C_j$, defined as 
the maximum priority among all tasks that access the resource. Whenever another 
task blocks on a locked resource, the owner's priority is temporarily raised to
the resource ceiling for the remainder of its critical section. This reduces 
context-switching overhead and simplifies the implementation in comparison to 
PIP.

IPCP is a variant of PCP, aiming for performance and ease of implementation. 
The major difference is that the task owning the resource has its priority 
raised to the ceiling immediately when it first acquires the resource, and not 
when another task tries to lock the resource. The main effect of this change is 
a further reduction of context switching overhead.

SRP provides resource access control for dynamic scheduling policies, such 
as the EDF scheduler. Additionally to a priority, SRP 
assigns a preemption level $\pi_i$ to each task. $\pi_i$ is static, initialized 
at the task $\tau_i$ creation time, and remains the same for all of its job. 
The main property is that a task $\tau_a$ can only preempt another task 
$\tau_b$ if its preemption level $\pi_a$ is greater than $\pi_b$. Under EDF 
scheduling, this condition is satisfied when preemption levels are ordered 
inversely in respect to the tasks' relative deadlines, as in $\tau_a > \tau_b 
\Longleftrightarrow D_a < D_b$.

During execution, every resource is assigned a resource ceiling $C_{R_i}$. This 
ceiling is equal to the maximum preemption level of the tasks that would be 
blocked on the resource, when issuing their maximum request. Therefore, the 
resource ceiling is a dynamic value, and is a function of available units
of the semaphore: $C_{R_i} = max(\pi_i|\mu_i(R_k) > n_k$.

SRP also defines a system ceiling $\Pi_s$, which is the maximum current system 
ceiling between all tasks. This is a global, dynamic parameter that can change 
at any resource access or release. The main idea of SRP is that if a task is 
going to be blocked on a resource it cannot access, it will be not even be 
allowed to preempt execution of other tasks. Also, a task is not allowed to 
begin execution unless any task that could preempt it would not block waiting 
on a resource. This is achieved by the SRP Preemption Test: a task is not 
permitted to be executed unless its priority is highest amongst all ready tasks, 
and its preemption level is higher than the system ceiling.

\section{Design and Implementation of Resource Access Protocols}
\label{sec:des}

Figure~\ref{fig:uml_class_sync} presents an overview of the proposed software 
design for the resource access protocols through a UML class diagram. To 
achieve this final design, we used the Application-Driven Embedded System 
Design (ADESD)~\cite{Froehlich:2001} methodology to capture common 
characteristics related to the PIP, PCP, IPCP, and SRP protocols. In the 
resulting design, there are in total eight classes: 
\texttt{Synchronizer\_Common}, \texttt{Semaphore}, \texttt{Semaphore\_RT}, 
\texttt{Semaphore\_SRP}, \texttt{Semaphore\_Ceiling}, \texttt{Semaphore\_PCP}, 
\texttt{Semaphore\_IPCP}, and \texttt{Semaphore\_PIP}. Below, we provide a 
detailed description of each class.

\Fig{uml_class_sync}{UML class diagram of the synchronization 
protocols.}{scale=.63}

The base class \texttt{Synchronizer\_Common} offers support for operations 
common to all synchronization primitives, such as mutexes, semaphores, and 
condition variables. These operations include, for instance, atomic increment 
and decrement (\emph{finc} and \emph{fdec}), test and set lock (\emph{tsl}), 
and interrupt enabling/disabling (\emph{begin\_atomic} and \emph{end\_atomic}). 
The class also implements an interface for common thread operations, such as 
sleep, wakeup, and wakeup all. These thread operations basically put a thread 
to sleep into the synchronization queue (the \texttt{\_queue} attribute), 
wakeup a thread that was sleeping in the queue, and wakeup all threads that 
were sleeping in the queue. All operations are protected, which means that they 
are only accessible by subclasses.

The \texttt{Semaphore} class implements the traditional \emph{p} and \emph{v} 
semaphore operations~\cite{Dijkstra:1968}. The class has an integer 
(\texttt{\_value}) as attribute, which is used to count the signals issued by 
the \emph{p} and \emph{v} (decrement and increment, respectively) through the 
\texttt{fdec} and \texttt{finc} operations implemented in the base class. It 
also uses the \texttt{sleep}, \texttt{wakeup}, and \texttt{wakeup\_all} 
operations from the base class.

The \texttt{Semaphore\_RT} class is common to all real-time resource access 
protocols. It has two attributes, \texttt{\_owner} and \texttt{\_priority} that 
represent, respectively, the current thread that owners the semaphore 
(\emph{i.e,} a thread that has entered a critical section through the \emph{p} 
operation) and the priority of that thread. The class offers public methods to 
set and get the attributes. Also, it has two protected methods,
\texttt{current\_thread} and \texttt{next\_thread}, that return the current 
thread being executed (note that it can be different from the \texttt{\_owner}) 
and the next thread that is the head of the semaphore's queue. These two 
operations are required by the PIP, PCP, and IPCP protocols. 

The \texttt{Semaphore\_PIP} class implements the priority inheritance 
behavior of the PIP protocol. It overwrites the \emph{p} and \emph{v} methods 
from the \texttt{Semaphore} to include the handling of the priority 
inheritance.  For the \emph{p} operation, the class first checks whether there 
is a thread inside the critical section by verifying if \texttt{\_owner} is 
zero. If so, the calling thread becomes the new semaphore's owner. If not, 
there is a test to check whether the calling thread priority is greater than 
the owner's priority. If it is, owner has its priority raised to the calling 
thread's priority. Then, the \emph{p} operation from the \texttt{Semaphore} is 
called to conclude the work. This is all done in an atomic state (\emph{i.e.,} 
interrupts are disabled). For the \emph{v} operation, if there is a semaphore 
owner, \texttt{Semaphore\_PIP} checks if there is another thread waiting on the 
Semaphore's queue to enter the critical section. If so, the \texttt{\_owner} 
and \texttt{\_priority} are updated. If not, \texttt{\_owner} and 
\texttt{\_priority} receives zero as value. Then, the \emph{v} operation from 
the \texttt{Semaphore} is called to conclude the work. The instruction within 
the \emph{v} are also performed with interrupts disabled.

The \texttt{Semaphore\_Ceiling} class is common to all ceiling-based protocols, 
such as IPCP and PCP. It adds an integer attribute (\texttt{\_ceiling}), which 
represents the semaphore's ceiling. It also has the set and get methods for the 
attribute. The \texttt{Semaphore\_PCP} class implements the PCP protocol by 
overwriting the \emph{p} and \emph{v} methods. The protocol raises the 
semaphore's owner thread priority to the ceiling whenever there is a call to 
\emph{p}. If the semaphore has no owner, then the current thread becomes the 
new owner and the semaphore's priority is set to the thread's priority. After 
that, the \emph{p} method of the \texttt{Semaphore} is called. The \emph{v} 
operation reestablishes the owner thread's priority (in case it was raised) and 
checks whether there is another thread waiting on the semaphore's queue. Then, 
it calls the \emph{p} method of the \texttt{Semaphore}. 

The \texttt{Semaphore\_IPCP} class implements the IPCP protocol, in a similar 
way to the \texttt{Semaphore\_PCP} class. The only difference is that the owner 
priority is raised to the semaphore's ceiling whenever a thread enters the 
critical section. Also, the owner priority is always reestablished to its 
original priority whenever it calls \emph{v}.

Finally, the \texttt{Semaphore\_SRP} class implements the Stack Resource Policy 
(SRP) protocol. It is statically configured (at compile time) to support a 
given number of tasks per resource, and also a set number of resources in the 
whole system. Every instance has a resource ceiling attribute 
(\texttt{\_ceiling}), which corresponds to the highest preemption level amongst 
the tasks that would block if accessing the resource. Whenever the semaphore 
value changes, the resource recalculates its resource ceiling and also the 
system ceiling. 

To make this dynamic accounting possible, the semaphore stores a list of tasks 
that access the resource, and the maximum number of resource units the task can 
hold at once (\texttt{\_prio\_levels} attribute). The system also needs to keep 
track of all existing SRP resources, and does so in a static array attribute 
(\texttt{\_resources}). 

\subsection{Implementation in an RTOS}

We have implemented the described protocols design in the Embedded Parallel 
Operating System (EPOS)~\cite{Froehlich:2001, epos}. EPOS is a
multi-platform, object-oriented, component-based, embedded system framework
implemented in C++. EPOS is the first open-source RTOS designed from scratch
that supports partitioned, global, and clustered versions of EDF, RM, LLF, and
DM scheduling policies~\cite{Gracioli:2013a}. A complete review of the real-time
support on EPOS can be found in~\cite{Gracioli:2013a}. We choose EPOS, because 
it supports static and dynamic scheduling and it is written in an 
object-oriented language. Thus, it is compatible with our proposed design. 
Moreover, EPOS until this work did not have a complete support for real-time 
resource access protocols. We believe that the proposed design can be replicated 
in any object-oriented RTOS written in C++ and that has EDF and RM schedulers.

Figure~\ref{fig:sequence_semaphore_pcp_p} shows the implementation of the 
\texttt{Semaphore\_PCP} \emph{p} operation. It uses the \emph{begin\_atomic} 
method from the \texttt{Synchronizer\_Common} to disable interrupts. Then, it 
uses the methods from \texttt{Semaphore\_RT} class to get the current running 
thread and modify or not the semaphore's ceiling. The \texttt{RT\_Thread} class 
represents a running thread and it provides two \emph{priority} methods to 
change and to get the real-time thread's priority. It is clear that code reuse 
is achieved by the inherent use of the class hierarchy. For instance, the 
\texttt{Semaphore} \emph{p} operation is called through the 
\texttt{Semaphore\_RT} class, in the operation number 15 in the sequence diagram 
class.  The other protocols operations use the same strategy to allow maximum 
code re-usability. 

\fig{sequence_semaphore_pcp_p}{UML sequence diagram of the 
\texttt{Semaphore\_PCP} \emph{p} operation.}{width=\columnwidth}

%Show an example of a schedule (EPOS TRACE VIEW)

\section{Experimental Evaluation}
\label{sec:eval}

This section describes the experimental evaluation of the previously described 
implementation. Our objectives are threefold: (i) to measure the memory 
consumption of the implementation; (ii) to evaluate the run-time overhead of 
the implementation; and (iii) to verify the impact of the run-time overhead 
into the system schedulability. The next subsections show the obtained results 
for the memory consumption, run-time overhead, and schedulability analysis, 
respectively.

\subsection{Memory Footprint}

For measuring the memory consumption of our implementation, we have executed 
EPOS on top of the EPOSMote III platform~\cite{epos}. EPOSMote III features an 
ARM Cortex-M3 32~MHz processor, with 32~Kb of RAM, and 512~Kb of flash. We used 
the \emph{GNU gcc} compiler for ARM at version 4.4.4 to generate the code. For 
measuring the memory footprint, we used the \emph{GNU objdump tool} available 
together with the gcc toolkit.

Table~\ref{tab:memory} shows the obtained memory footprint for each class (see 
Figure~\ref{fig:uml_class_sync}). The table also shows the total lines of code, 
just to correlate the memory footprint with the implementation. Memory usage is 
split into code, data, and static data sections. Data represents the memory 
consumed by an instance of the corresponding semaphore type, not including the 
footprint of the base classes. The total memory consumption describes the memory 
consumed by the implementation of the semaphore subclass and a single 
corresponding object instance. For instance, the total memory consumption of 
the \texttt{Semaphore\_PIP} is 296~bytes, which represents its own 140~bytes 
summed with \texttt{Semaphore} and \texttt{Semaphore\_RT} memory consumptions. 
It is important to highlight that code from the \texttt{Synchronizer\_Common} 
class is not represented in the Table, because it is inlined into the methods 
that use the base class. It means that its code is implemented in the header 
file to improve the run-time overhead by avoiding explicitly method calls. 

\begin{table}[!ht]
\centering
\caption{Memory footprint (in bytes) and lines of code of the implemented 
classes.}
\begin{tabular}{l l l l l|l l}
% \cline{1-7}
Class	& Code &  Data & Static Data & Total Mem. & 
\multicolumn{2}{c}{Lines of Code} \\ 
			&	   &	   &			 &	Consum.	 & Header & Source \\
\hline
Semaphore 			& 132 & 16 & 0  & 148 & 10 & 10 \\ \hline
Sem. RT		 		& 0   & 8  & 0  & 8   & 24 & 0  \\ \hline
Sem. Ceiling	 	& 0   & 4  & 0  & 4   & 9  & 0  \\ \hline
PIP 				& 140   & 0  & 0  & 140   & 8  & 35 \\ \hline
PCP 				& 144  & 0  & 0  & 144  & 8  & 28 \\ \hline
IPCP 				& 144  & 0  & 0  & 144  & 8  & 28 \\ \hline
SRP 				& 368 & 64  & 68 & 504 & 78 & 18 \\ \hline
\end{tabular}
\label{tab:memory}
\end{table}

\subsection{Run-time Overhead}


For measuring the run-time overhead of the implementations, we used the  
EPOSMote’s 32~MHz timer with ±40ppm accuracy. For each protocol, the test 
consisted of a set of 20 tasks that would try to acquire the same resource in a 
cascade, and subsequently release the resource. Then the relevant code sections 
in the OS were instrumented to account for their execution time. Every 
\textit{p} and \textit{v} method of every semaphore type was instrumented, as 
well as the system code responsible for queuing and dequeuing tasks, which is 
used by the base semaphore implementation. These sections were timed 
with a small helper utility, that wraps the code, setups a hardware timer 
peripheral, and uses it to count the amount of time consumed in system clock 
cycles by the code section. We repeated the test 10 times and extract the 
worst-case run-time overhead from these executions.

Figure~\ref{fig:run_time_overhead_eposmote} presents the obtained 
worst-case run-time overhead for each protocol in~\si{\micro\second} to perform 
\textit{p} and \textit{v} operations, as a function of the number of tasks 
waiting on the semaphore. It was observed that for \textit{p} operations of the 
PIP, IPCP, and PCP protocols, the overhead depends significantly on the number 
of tasks currently waiting on the semaphore. This is a consequence of queuing 
tasks when they block waiting on the resource. This queue is implemented as a 
linked list, and as the queue grows larger, the enqueuing overhead increases. 
This queue overhead overcomes the overhead of the protocols' code. However, we 
can note a very small difference when there is no tasks waiting on the 
semaphore. In that case, IPCP has a smaller overhead (almost imperceptible in 
the graph).

\fig{run_time_overhead_eposmote}{The obtained run-time overhead running the 
implemented protocols on the EPOSMote III platform.}{width=\columnwidth}

In contrast, the overhead for the \textit{v} operation does not change with the 
number of tasks in the queue, because dequeuing is always done from the head of 
the queue, so this is a constant-time operation. PCP and IPCP have almost the 
same overhead for the \emph{v} operation, because both protocols do similar 
operations. The overhead for PIP, however, is a little bit larger (up to 
1.2~\si{\micro\second}) due to its more sophisticated operations. The \emph{v} 
operation of the SRP is the worst, because it demands an updating of the system 
ceiling, which is performed in a loop (its size is equal to the number of 
tasks that use the resource, 20 in our experiments). 

The case of tasks being blocked on a semaphore under SRP should never happen in 
principle. This is because of the preemption test, that disables a task from 
beginning execution if one of its resource accesses would cause it to block. 
However, in our practical implementation, non-real-time tasks do not have an 
assigned preemption level, and are unaffected by SRP. Therefore, in the case of 
a soft real-time application that makes mixed use of tasks types, tasks may 
still block on a resource.

\subsection{Schedulability Impact}

For measuring the impact of the run-time overhead into the schedulability of 
task sets, we used a methodology similar to the one proposed 
by Yang et al~\cite{Yang:2015}, adapting their 
methodology to our scenario (\emph{i.e,} uniprocessor system). We generated 
several task sets according to the following rules. A task's period $p_i$ was 
randomly chosen from a log-uniform distribution ranging over 
[10\si{\milli\second}, 100\si{\milli\second}] (\emph{homogeneous periods}) or 
[1\si{\milli\second}, 1000\si{\milli\second}] (\emph{heterogeneous periods}). 
Each task's utilization $u_i$ was randomly chosen from a uniform distribution 
ranging over [0.05\si{\milli\second}, 0.1\si{\milli\second}] (\emph{light 
utilizations}) or [0.1\si{\milli\second}, 0.25\si{\milli\second}] 
(\emph{medium 
utilizations}), and the task's WCET $e_i$ was set to $e_i = p_i \times u_i$. 
The number of shared resources was varied across $n_r \in \{1, 2, 4, 8\}$. 
For each resource, there is a probability $p^{acc}$ associated with each task 
to access the respective resource. We vary the probability across $p^{acc}$ 
$\in$ \{0.1, 0.25, 0.5\}. The maximum critical section length $L_{i,q}$ was 
chosen uniformly from [1\si{\micro\second}, 25\si{\micro\second}] 
(\emph{short}), [25\si{\micro\second}, 100\si{\micro\second}] (\emph{medium}), 
and [100\si{\micro\second}, 500\si{\micro\second}] (\emph{long}).  Each task's 
job uses a resource only once per activation ($N_{i,q}$ = 1). 

For each configuration (144 combinations\footnote{Due to space constraint, 
we only show the graphs of 6 combinations in the paper. All graphs will be 
available at \url{http://epos.lisha.ufsc.br}.}), we generated 1000 
task sets per utilization cap, varying it from 0.1 to 1. For each task set and 
utilization cap, we first applied the traditional PIP, PCP, and SRP 
schedulability analyses~\cite{Buttazzo:2011} without considering any run-time 
overhead. IPCP has the same schedulability test as PCP~\cite{Buttazzo:2011}. For 
PIP, PCP, and IPCP we consider the RM scheduling, while for SRP we consider the 
EDF scheduling, to evaluate both static- and dynamic-priority scheduling 
approaches. Then, we inflated each task's WCET $e_i$ with the obtained 
run-time overhead (see Figure~\ref{fig:run_time_overhead_eposmote}). To obtain 
the correct run-time overhead, we used the following approach. Considering a 
task 
$\tau_i$ and a task set with $n$ tasks. For each resource $R_j$ used by 
$\tau_i$, we iterated over the tasks finding the other tasks that also use 
$R_j$. Then, we obtained the maximum number of tasks that share the same 
resources with $\tau_i$. This gave us the worst-case number of tasks that can 
wait on the same resource. We used this number to take the overhead. We 
implemented the task generation methodology as well as the protocol 
schedulability analyses~\cite{Buttazzo:2011} in the SchedCAT 
tool~\cite{schedcat}\footnote{The scripts used during the experiments are 
available online at \url{http://epos.lisha.ufsc.br.}}. 

Due to the large combination of configurations, we only focus here on the major 
findings. Figures~\ref{fig:label-a}--\ref{fig:label-f} shows the 
representative graphs. On the \emph{x-axis} we vary the utilization cap of the 
generated task sets. On the \emph{y-axis} we vary the schedulability ratio. For 
instance, a schedulability ratio of 0.6 means that 60\% out of the 1000 task 
sets were schedulable. Below, we discuss the main observed behaviors from the 
obtained results.

\begin{figure*}
\centering
\begin{minipage}[b]{.48\textwidth}
\includegraphics[width=\textwidth]{fig/{light_heterogeneous_long_4_0.25}.pdf}
\caption{Heterogeneous periods, light utilizations, long critical 
sections, $n_r = 4$, $p^{acc} = 0.25$, and $N = 1$.}\label{fig:label-a}
\end{minipage}\qquad
\begin{minipage}[b]{.48\textwidth}
\includegraphics[width=\textwidth]{fig/{light_heterogeneous_medium_4_0.25}.pdf}
\caption{Heterogeneous periods, light utilizations, medium critical 
sections, $n_r = 4$, $p^{acc} = 0.25$, and , $N = 1$.}\label{fig:label-b}
\end{minipage}\qquad
\begin{minipage}[b]{.48\textwidth}
\includegraphics[width=\textwidth]{fig/{light_heterogeneous_short_4_0.25}.pdf}
\caption{Heterogeneous periods, light utilizations, short critical 
sections, $n_r = 4$, $p^{acc} = 0.25$, and , $N = 1$.}\label{fig:label-c}
\end{minipage}\qquad
\begin{minipage}[b]{.48\textwidth}
\includegraphics[width=\textwidth]{fig/{light_heterogeneous_long_8_0.5}.pdf}
\caption{Heterogeneous periods, light utilizations, long critical 
sections, $n_r = 8$, $p^{acc} = 0.5$, and , $N = 1$.}\label{fig:label-d}
\end{minipage}\qquad
\begin{minipage}[b]{.48\textwidth}
\includegraphics[width=\textwidth]{fig/{light_homogeneous_long_8_0.5}.pdf}
\caption{Homogeneous periods, light utilizations, long critical 
sections, $n_r = 8$, $p^{acc} = 0.5$, and , $N = 1$.}\label{fig:label-e}
\end{minipage}\qquad
\begin{minipage}[b]{.48\textwidth}
\includegraphics[width=\textwidth]{fig/{medium_homogeneous_long_8_0.5}.pdf}
\caption{Homogeneous periods, medium utilizations, long critical 
sections, $n_r = 8$, $p^{acc} = 0.5$, and $N = 1$.}\label{fig:label-f}
\end{minipage}
\end{figure*}

\textbf{Efficient implementation provides schedulability bounds close to the 
theoretical ones.} Our results indicate that proper implementation of resource 
access protocols provide small impact on the schedulability ratio of task sets 
in most cases. For instance, in a scenario with high demand for shared 
resources (Figures~\ref{fig:label-d}--\ref{fig:label-f}), the difference 
between the schedulability ratio with and without overhead is less than 5\% in 
all scenarios. An exception is when the utilization cap is between 0.9 and 1, 
where the overhead has an important influence on the schedulability, because 
the utilization of the task sets are very close to the scheduling bounds (see 
Figure~\ref{fig:label-b} for example). 

\textbf{Critical section sizes heavily impact schedulability for 
light-utilization task sets.} As expected, Figures~\ref{fig:label-a}, 
\ref{fig:label-b}, and~\ref{fig:label-c} show that for task sets with a great 
number of low utilization tasks, critical section sizes have a strong impact on 
the schedulability ratio. We can note that with longer critical sections, the 
schedulability is reduced. This is a consequence of blocking time bounds for all 
protocols being derived directly from critical section lengths.

\textbf{Homogeneity of task periods has a major effect on the schedulability ratio.}
For a given total utilization, heterogeneous task sets contain tasks with 
shorter periods. Since critical section lengths for a given resource are the 
same for all tasks, short period tasks were more affected by blocking-time 
bound terms on the schedulability analyses. Because of this, the schedulability 
ratios for task sets with heterogeneous periods were lower (see 
Figures~\ref{fig:label-d} and~\ref{fig:label-e} for instance).

\textbf{Schedulability is higher in task sets with medium utilizations.} This 
is mainly due to the fact that task sets with medium utilizations have fewer 
tasks when compared to light utilizations (compare Figures~\ref{fig:label-e} 
and~\ref{fig:label-f}). It can also be observed in Figure~\ref{fig:label-e} that 
for a higher number of tasks, if periods are homogeneous, PCP performs much 
better than PIP.

\textbf{Probability of accessing a resource impacts the schedulability of task 
sets considering overhead.} When we used higher probabilities to access shared 
resources, the schedulability of task sets considering the run-time overhead 
decreased, as expected. This behavior can be clearly seen in 
Figures~\ref{fig:label-a} and~\ref{fig:label-d}.

\textbf{Ceiling-based protocols are better than the PIP.} In all tests, the 
ceiling-based protocols, PCP and IPCP, have presented better schedulability 
ratios than PIP. This is also expected, as the PCP/IPCP have lower blocking 
times given by their schedulability test when compared to the PIP test. PIP has 
shown to be a little bit less sensitive to the run-time overhead than PCP and 
IPCP.

\section{Related Work}
\label{sec:rel}

Suspension-based resource access protocols for uniprocessor real-time systems 
were first proposed by Sha et al~\cite{Sha:1990}. The authors have proposed PCP 
and PIP and the work has served as basis to many other researches. SRP was 
proposed by Baker in 1991~\cite{Baker:1991} and it was also a seminal work, 
mainly for providing resource access protection for dynamic scheduling. 

Several general-purposed OSes and RTOSes implement some of the 
analyzed real-time resource access protocol. For instance, FreeRTOS employs 
PIP in the mutex primitive~\cite{freertos} and also supports 
SRP~\cite{Inam:2011}. $LITMUS^{RT}$ supports PCP, SRP, and several 
multiprocessor resource access protocols~\cite{Brandenburg:2008,Spliet:2014}. 
The L4 microkernel~\cite{Liedtke:1995} and Linux support priority inheritance. 
Linux also implements IPCP, under the name PRIO\_PROTECT in the pthreads 
library. Lee and Kim implemented PIP in the \si{\micro}C/OS-II 
kernel and measured the run-time overhead running the 
implementation on top of the CalmRISC16 evaluation board~\cite{Lee:2003}. The 
observed run-time overhead for the \emph{p} and \emph{v} semaphore operations 
was 30.5~\si{\micro\second}. Researchers also proposed to move the mechanisms 
to control the priority inheritance~\cite{Akgul:2003} or the semaphore 
structures~\cite{Marcondes:2009} to the hardware in order to reduce the run-time 
overhead.

Wang et al. implemented multi-resource versions of PIP and PCP 
in a component-based OS for controlling the access to shared 
stacks~\cite{Wang:2011}. In their experimental evaluation considering the 
schedulability of generated tasks, PIP has performed better than PCP. Thus, 
they have concluded that PIP has potential to provide a high-degree of 
schedulability~\cite{Wang:2011}. In our experimental evaluation, however, PIP 
has presented similar performance in terms of overhead when compared to PCP, but 
had worse schedulability ratios.

Although not directly involved with this paper, resource access protocols for 
multiprocessors have been the subject for many researchers recently. 
MPCP~\cite{Rajkumar:1990} and MrsP~\cite{Burns:2013} are two multiprocessor 
variants of the PCP. Flexible Multiprocessor Locking Protocols (FMLP) is a 
collection of protocols for global and partitioned scheduling~\cite{Block:2007}. 
It was designed to efficiently deal with short non-nested access and to allow 
unrestricted critical section nesting. Parallel Priority Ceiling Protocol 
(P-PCP) extends PIP to avoid unfavorable blocking situations, but increases the 
run-time overhead~\cite{Easwaran:2009}. Biondi and 
Brandenburg~\cite{Biondi:2016} have revisited four synchronization protocols 
under partitioned EDF (P-EDF) scheduling and compared them in terms of 
schedulability. They concluded that the lock-free synchronization approach 
offers advantages on asymmetric multiprocessing platforms. Yang et al. proposed 
new schedulability analyses based on linear programming for several 
multiprocessor semaphore-based locking approaches~\cite{Yang:2015}. The authors 
claim that the new analyses are more accurate than prior approaches.   
\section{Conclusion}
\label{sec:conc}

This paper presented an object-oriented design of four uniprocessor real-time 
resource protocols (PIP, PCP, IPCP, and SRP). We have implemented the proposed 
design in an RTOS and measured the memory footprint and run-time overhead of 
the implementation. By compiling and running our implementation in a 
microcontrolled platform, we proved that it allows small memory footprint (from 
296 to 660 bytes, depending on the protocol) and run-time overhead (less than 
20~\si{\micro\second} for 20 tasks). Moreover, we used the run-time overhead to 
analyze how it affects the system schedulability and proved that efficient 
RTOS implementation of uniprocessor resource access protocols has few impact on 
the system schedulability. 

As future work, we plan to extend the design to support multiprocessor 
resource access protocols, such as MPCP and MSRP, then implement these 
protocols in EPOS and repeat the schedulability analysis considering the 
run-time overhead.


\bibliographystyle{IEEEtran}
\bibliography{references}

% that's all folks
\end{document}


